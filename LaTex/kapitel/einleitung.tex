% einleitung.tex
\chapter{Introduction}
%\begin{enumerate}
%
%    \item about hashfunctions
%    \item Definition of md5
%    \item my impelmentation of md5
%        \begin{enumerate}
%            \item padding (missing in Stevens code )
%            \item md5compress
%            \item potential for improvement 
%        \end{enumerate}
%    \item collisions for md5
%        \begin{enumerate}
%            \item Blocks 1 and 2 finding 
%
%            \item MMM
%        \end{enumerate}
%    \item conclusion
%    \item Definition of SHA1
%    \item why not in SHA1
%    \item notation helper
%
%\end{enumerate}
\begin{table}[]
    \caption*{Pages}
    \begin{tabular}{ccccc}
    Chapter & Pages & \%    & content   &  \\
        1   &   2   &   5   & Intro     &  \\
        2   &   5   &   14  & Basic MD5 &  \\
        3   &   6   &   20  & coll-theroy & \\
        4   &   9   &   30  & colls    & \\
        5   &   4   &   14  & Othercolls& \\
        6   &   4   &   14  & SHA1      & \\
        7   &   1   &    3  & Outro     & \\
      A-D   &   0   &    0  & Additions & \\
      sum   &$\sim$30 & $\sim$ 100 &    & \\
    \end{tabular}
    \end{table}

This thesis is about a deeper look on MD5.
We take a closer look at the Master thesis of M. Stevens: a fast collisions finding algorithm [1].
The goal is to work out a more clear and understandable code, which is not necessarily faster, to reevaluate the code on modern systems and the difference to SHA1. 

\section{Hash Function}
A \textit{Hash Function} is something
\section{Notation / Preliminaries}

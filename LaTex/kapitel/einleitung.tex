% einleitung.tex
\chapter{Introduction}
\begin{enumerate}

 
    \item Introduction
    \item about hashfunctions
    \item about md5
    \item about collisions
    \item goals and motivation
    \item notation explained
    \item Definition of md5
    \item impelmentation of md5
        \begin{enumerate}
            \item padding (difficulties, missing by stevens )
            \item processing
            \item md5compress
            \item representation
            \item potential for improvement 
            \item difficulties
        \end{enumerate}
    \item attack of md5
\end{enumerate}
\section*{Motivation}
This thesis is about a deeper look on MD5 and SHA1.
We take a closer look at the Master thesis of M. Stevens: a fast collisions finding algorithm [1].
The goal is to work out a more clear and understandable code, which is not necessarily faster, to reevaluate the code on modern systems and the difference to SHA1. 


\section*{nodes}
 
Bsp für Stevens Werte für $Q_t$ mit $t = 3$ :\\

Bit Conds for $Q_t | t =3$ :

$4.$ the old bit Conds\\
$3.$ the new bit Conds\\
$2.$ the val to set the ones ( or with $0x017841c0$)\\
$1.$ the val to set the zeros (and with 0xfe87bc3f)\\

\begin{align*}    
    1.& \& & 11111110& &10000111& &10111100& &00111111& &0xfe87bc3f \\
    2.& \| & 00000001& &01111000& &01000001& &11000000& &0x017841c0 \\
    3.&    & ........& &.1111...& &.1....01& &11......& & \\
    4.&    & ........& &....0...& &....0...& &.0......& & 
\end{align*}

  the $\&$ flips the $0$ correct, the $\|$ flips the $1$ correct


  \newpage

\section{Notation helper}


\begin{table}[]
    \begin{tabular}{ c | c | c }
    Stevens & Wang  & Definition  \\
    \hline 
    $ RL \left(X , Y \right) $  & $ ROTL^{Y} \left( X\right) $  & cyclic left shift $X$ by $Y$ (usually mod 31) \\
    $ RL \left(X , Y \right) $  & -                             & cyclic right shift $X$ by $Y$  \\
    $ RC \left(t \right) $      &$ S \left(t \right) $          & rotation Constant of $t$ \\
    Block 1, Block 2 & Block N, Block M & pair of first blocks for collisions finding   \\
    Block 0, Block 1 & Block N, Block M & same pair but im Code\\
    \end{tabular}
\end{table}


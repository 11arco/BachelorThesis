% einleitung.tex
\chapter{Introduction}
\section{May Contains}
\begin{enumerate}

    \item about hashfunctions
    \item Definition of md5
    \item my impelmentation of md5
        \begin{enumerate}
            \item padding (missing in Stevens code )
            \item md5compress
            \item potential for improvement 
        \end{enumerate}
    \item collisions for md5
        \begin{enumerate}
            \item Blocks 1 and 2 finding 
            \item 00, 01, 10, 11
            \item MMM
        \end{enumerate}
    \item conclusion
    \item Definition of SHA1
    \item why not in SHA1
    \item notation helper

\end{enumerate}
\section*{Motivation}
This thesis is about a deeper look on MD5.
We take a closer look at the Master thesis of M. Stevens: a fast collisions finding algorithm [1].
The goal is to work out a more clear and understandable code, which is not necessarily faster, to reevaluate the code on modern systems and the difference to SHA1. 


\section*{Notes}




\section{Notation Helper and Ideas}


\begin{table}[]
    \caption*{\large Notation}
    \begin{tabular}{ c | c | c }
    Stevens & Wang  & Definition  \\
    \hline 
    $h(m) = H$ &  & the hash $H$ of some message $m$\\
    $ RL \left(X , Y \right) $  & $ ROTL^{Y} \left( X\right) $  & cyclic left shift $X$ by $Y$ (mod 31) \\
    $ RR \left(X , Y \right) $  & -                             & cyclic right shift $X$ by $Y$ (mod 31) \\
    $ RC \left(t \right) $      &$ S \left(t \right) $          & rotation Constant of $t$ \\
    Block 1, Block 2 & Block N, Block M & pair of first blocks for collisions finding   \\
    Block 0, Block 1 & Block N, Block M & same pair but im Code\\
    \end{tabular}
    \label{notation}
\end{table}

$RR\left(X,Y\right) :\equiv X  \circled{>>} Y$\\
$RL\left(X,Y\right) :\equiv X  \circled{<<} Y$\\
$X \oplus Y :\equiv X "XOR"  Y$


\begin{align*}
10000000000000000000000000000001  \circled{<<} 7 &\equiv 00000000000000000000000011000000
\end{align*}
  



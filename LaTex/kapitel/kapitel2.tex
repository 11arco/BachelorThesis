% kapitel2.tex
\chapter{About MD5}
\label{chapter:kap2}
\section{Definition}
MD5 stands for \textit{Message-Digest Algorithm 5} since it generates a digest for any given text. MD5's output length is $128$ Bit, represented in hexadecimal. Since the amount of possible text it close to
and the amount of MD5s is limited by $32^{16}$, some text may have the MD5. If two different inputs create have the same hash value, that is what we call a collision.
MD5 is usually seen as 4 steps as seen in fig {md5}\\
\begin{figure}
    \begin{enumerate}
        \item padding
        \item procssesing
        \item md5 sum 
        \item output
    \end{enumerate}
\caption{MD5 algo}
\label{md5}
\end{figure}

\section{On Code}

As mention before, we can describe the MD5 hash algorithm in four steps.
The implementation of the padding is the least interesting, yet there are many mistakes one can make.
To archive an adequate comparison we need build simply ignore the process of padding and pretend to expect the padding to be done, even thou we do it ourselves.
The processing step is were the codes vary the most. Stevens works a lot with global variables. All in all his code is extremely optimized for performance.

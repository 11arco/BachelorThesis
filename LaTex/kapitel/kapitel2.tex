% kapitel2.tex
\chapter{Implementation of MD5}
\label{chapter:kap2}
\section{MD5}
MD5 stands for \textit{Message-Digest Algorithm 5} since it generates a digest for any given text. MD5's output length is $128$ Bit, represented in hexadecimal. Since the amount of possible text it close to
and the amount of MD5s is limited by $32^{16}$, some text may have the MD5. If two different inputs create have the same hash value, that is what we call a collision.
MD5 is usually seen as 4 steps as seen in fig {md5}\\
\begin{figure}
    \begin{enumerate}
        \item padding
        \item procseesing
        \item md5 sum 
        \item output
    \end{enumerate}
\caption{MD5 algo}
\label{md5}
\end{figure}
% kapitel3.tex
\chapter{ Collision Finding}
\label{chapter:kap3}
\section {About collisions }
Hash functions have resistance:
\begin{enumerate}
    \item First Preimage Resistance: for a hash function $h(m) = H$ the massage $m$ is hard to find.       
    \item Second Preimage Resistance: for a given messages $m_1$ it is hard to find an $m_2$ with $m_1 \neq m_2 $ and $h(m_1) = h(m_2)$.
    \item Collision Resistance: two arbitrary messages  $m_1$ and $m_2$ with $\neq m_2 $ and  $h(m_1) = h(m_2)$ are hard to find.
\end{enumerate}


\section{Differential Path}
Stevens starts with Wangs attack, which trys to finde to pairs of blocks:
\( \left(B_0 , B_0' \right) \)  and \( \left(B_1 , B_1' \right) \) that \( IHV = IHV' \),
 with the goal to create two massages $ M $ and $ M' $, with the same hash value:

\begin{align*}
    &IHV_0 &\xrightarrow[M_{(1)}] {}&\cdots &\xrightarrow[M_k]{} &IHV_k \xrightarrow[B_0]{} &IHV_{k + 1}  &\xrightarrow[B_1]{} &IHV_{k + 2}  &\xrightarrow[M_{k+1}]{}&\cdots &\xrightarrow[M_N]{} &IHV_N\\
    &=     &                        &       &                    &=                         &\ne          &                    &=            &                       &       &                    &= \\
    &IHV_0 &\xrightarrow[M_{(1)}] {}&\cdots &\xrightarrow[M_k]{} &IHV_k \xrightarrow[B_0]{} &IHV'_{k + 1} &\xrightarrow[B_1]{} &IHV'_{k + 2} &\xrightarrow[M_{k+1}]{}&\cdots &\xrightarrow[M_N]{} &IHV_N\\
\end{align*} 
The idea to manipulate a block $B$ such that $Q_1 \dots Q_{16}$ maintain their conditions and that $Q_17$ to some $Q_k$ do not change at all. We try to make k as large as possible.
\section{Bit Conditions}
Bit conditions describe the differential path on bits. 
We need the bit conditions to avoid a carry, so a manipulation in step $t$ stays in step $t$ and does not propagate beyond the $31$st bit.
We look at conditions and restrictions. The restrictions leads to conditions, which we calculate in the following.
A restriction e.g. $\Delta T_2 \left[\ 31 \right] = +1 $ leads to conditions $ Q_1\left[ 16 \right] Q_2\left[ 16 \right] = Q_3\left[ 15 \right] = 0 $ and $ Q_2[15] = 1$.
Notice, conditions are on $\Delta T_t[i]$ a state in md5-algorithm before the rotation and restrictions are on $Q_t[i]$ states of the md5-algorithm after the rotation.\\ 
We calculate the bit conditions by using the Add-Difference for two massage blocks containing tow blocks $N|M$ and $N'|M'$. The XOR-Difference is useful, too.
\begin{align*}
    \delta X &= X' - X \left( mod32 \right) \text{ Add-Difference}\\
    \Delta X &= X' \oplus X \text{ XOR-Difference}\\
    \lambda \left[i\right] &= \text{ our guess for the ith bit: } X\left[i\right] \\
    \\
    \text{if } \Delta X = \lambda &\Rightarrow \delta x \text{ can be determined}
\end{align*}
For $\lambda \left[i\right]$ we only need to consider $i < 31$, since $X \left[ 31 \right]$ as msb always creates a add difference of $2^{31}$.\\
\begin{wrapfigure}[]{r}{0.2\textwidth}
    %\caption*{\textcolor{red}{fix writing}}
    %\begin{table}[]
        \begin{tabular}{| c | c |}
            \hline
            $t$ & $RC(t)$    \\
            \hline
            \hline
            0  & 7 \\
            1  & 12 \\
            2  & 17 \\
            3  & 22 \\
            4  & 7 \\
            5  & 12 \\
            6  & 17 \\
            7  & 22 \\
            8  & 7 \\
            9  & 12 \\
            10 & 17 \\
            11 & 22 \\
            12 & 7 \\
            13 & 12 \\
            14 & 17 \\
            15 & 22 \\
            \hline
        \end{tabular}
        \label{RC}
    %\end{table}
    \end{wrapfigure}
We calculate a $\delta$ for each $f_t$, $Q_t$, $T_t$ and $R_t$ for our add difference, to calculate $Q_{t+1}$ .
Additional we need the rotation constant $RC$ for each $t$. 
In general we begin with the $f_t$ since we want $f_t$ to be in a particular state.
Since we want to avoid a carries in our calculation 


\begin{align*}
    \delta F_t &= f \left( Q'_t,Q'_{t-1},Q'_{t-2} \right) \\
    \delta T_t &= \delta F_t + \delta  Q'_{t-3} + \delta W_t \\
    \delta R_t &=  RL \left(T'_t,RC_t \right) - RL \left(T_t,RC_t \right)\\
    \delta Q_{t+1} &= \delta Q_t + \delta R_t
\end{align*}


\begin{enumerate}
    \item $t \in \{0,1,2,3\}$:\\
     $Q_t = 0 $ since here is no influence by an message and no calculation of f, there is nothing to change:
    \item $t = 4$\\
    $\Delta T_4 = -2^{31}$, because we must not have a carry, we \textit{lock} the last bit.
    Since  $RL(T_4, RC_4) = RL(-2^{31}, 7) = -2^6 $ and $\delta Q_4 = 0 \Rightarrow  \delta Q_5 = -2^6$  
\end{enumerate}

\newpage


\begin{align*}
    m_t = RR \left( Q_{t+1} - Q_t , RC_t\right) - f_t \left( Q_t, Q_{t-1}, Q_{t-2} \right) - Q_{t-3} - AC
\end{align*}


\documentclass[aspectratio=169]{beamer}
\usepackage{graphicx}% Including graphics
\setbeamertemplate{footline}[frame number]
\newcommand{\tabitem}{%
  \usebeamertemplate{itemize item}\hspace*{\labelsep}}
  
\usepackage[german]{babel}
\usepackage{caption}
\usepackage{tikz}
\usepackage{wrapfig,lipsum,booktabs}
\usetikzlibrary{arrows,automata}

\title{Paralles-Hashing\\}
\subtitle{\textcolor{gray}{\scriptsize { Sequential and Parallel Algorithms and Data Structures
	von
	Peter Sanders, Kurt Mehlhorn, Martin Dietzfelbinger, Roman Dementiev}}}

\author{Marco Bellmann}
\begin{document}

\maketitle
\begin{frame}[t]{Lineares Hashing}
	\begin{center}
	\begin{figure}[!h]
 		\begin{tikzpicture}[->,>=stealth',shorten >=1pt,auto,node distance=4.0cm, semithick]
 			\node (1) [draw, circle, minimum size = 30mm] at (0, 0) {}  (2,0) node[right] {Schlüsseluniversum mit Teilmenge $S$};
 			\node[state]         	(3) 			{S };
 			\pause
  			\node(2)  [draw, rectangle,minimum width=3cm,  minimum height = 1.5cm] at(0,-3){$\vdots$} (0,-2.5)node{t}  (2,-3) node[right] {Hashtabelle };
  			\path (1)	edge 			node			{h}(2)(2,-2) node[right] {Hashfunktion };	

		\end{tikzpicture}
		\\
		\vfill

	  \caption{\scriptsize \textcolor{gray}{MIT 6.006 Introduction to Algorithms, Fall 2011}}


	\end{figure}

	\end{center}
\end{frame}

\begin{frame}[t]{Datenstruktur }	
%Hasgtabelle als Datenstruktur 
	\center

	
	Hashtabelle
	\pause
	\vfill
   	\begin{tabular}[h]{c|c c}
		\hline
		Funktionen 			&Sequentiel& \\
						& Bedingung & Aktion \\
		\hline
		find(x)      			& $x = key(e)  $	& return e or return$\bot$   \\
		insert(e)			& $e \not \in t$ 	& -  \\
		remove(x)			& $e \in t$		& -  \\

		\hline
	\end{tabular}
	\vfill
	\pause
   	$Element \: e :=(Key \: x ,Value \: v)$
	 
\end{frame}
\begin{frame}[t]{Arten von Hashing}
	\center
	\begin{tabular}[h]{c |c}
		Offenes Hashing		 	&	 Geschlossenes Hashing \\
		Linear Probing			&	 Verkettetes Hahing   \\
		\hline
		\pause
		Elemente können an einer	 	&	fixer Adressraum \\
		anderen Adresse liegen		&	Erweiterung mit Liste\\
		\pause
							\\
		Suche in ganzer Hashtabelle	&	Suche in Liste \\
		(Grafik)				& 	(Grafik)
		
	\end{tabular}
	 
	 \hfil
	 \pause
	 \\
		Suche: $O(n)$
	
 
\end{frame}

\begin{frame}[t]{Parallel vs Sequentiell}%Funktionsvergleich


	%echte Tabelle 
	%|		| find()	|delete()	|		|
	%|sequnziell	|arbeitsweise|
	%|		|return	|	
	%|parallel	|		|
   	%|		|		|
	
  	\vfill
  	 	
   	\begin{tabular}[h]{c|c c|c c}
		\hline
		Funktionen 			&Sequentiel& & Parralell& \\
							& Bedingung & Aktion& Bedingung &Aktion   \\
		\hline
		find(x)      				& $x = key(e)  $&{\small return e or $\bot$ }&$ x=key(e)$ &{\small return e.copy() or $\bot$ } \\
		insert(e)				& $e \not \in t$ & - & $e \not \in t$  & -  \\
		remove(x)				& $e \in t$& - & $e  \in t$ &  - \\
		\hline
		\pause
		update(x,v,f)			& - & - & $(x,v') \in t$ & $f ( v',v) $ \\
		{\small insertOrUpdate(x,v,f)}	& - & - & $e \not \in t$ & {\small $insert(x)$ or $f (v' , v)$ } \\
		\hline

	\end{tabular}
	\\
	\begin{flushright} {\scriptsize$Element \: e :=(Key \: x ,Value \: v)$} \end{flushright} 
	\pause
	\vfill
	Achtung:\\
	remove und insert nur einmal
	% muss aber nicht, wenn die Aufgaben so verteilt werden, dass jedes PE einen Teil des Datenstatzes einsortiert bei gemiensamer Tabelle
\end{frame}

\begin{frame}[t]{Distributed-Memory Hashing }

		
	\begin{center}
	Simuliere mehrere Tabellen\\
	\pause
	\textbf{PEs Adressraum:}\\
	für ein$i \in0\hdots p-1 $\\

		
	\hfill
	\pause
	
		\begin{figure}[!h]

 		\begin{tikzpicture}[->,>=stealth',shorten >=1pt,auto,node distance=4.0cm, semithick]%geschwungender pfeil
  			\node(1)  [draw, rectangle,] at(0,0){
  				
  				\begin{tabular}{c| c c}
                				
                				
                				0& &	\\
                				\hline
                				& &	\\
                				\vdots&  &	\\
                				& &	\\
                				\hline
                				m-1& &	\\
              		  		
            			\end{tabular}
  				  			
  			}  ;
 			\node(2)  [draw, rectangle,minimum width=1cm,  minimum height = 1.5cm] at(4,0){$0$} (4.6,0) node[right] {$\hdots$};
  			\node(3)  [draw, rectangle,minimum width=1cm,  minimum height = 1.5cm] at(6,0){$p-1$};
  			
  			\path (5,-0.22)	edge [bend left =60] 			node			{}(1.1,-0.22);	

		\end{tikzpicture}
		\\
		\pause
		\vfill
		$ i* m/p \hdots (i + 1)* m/p$\\
		\label{fig:f1}
	\end{figure}

	\end{center}

\end{frame}
\begin{frame}[t]{Beispiel }
	\center
		$ i* m/p \hdots (i + 1)* m/p$\\
		$m=40$, $p=5$\\
		\pause
		$\Rightarrow $ für $i=0$, Adressraum von $0$ bis 8\\
		
		\vfill
	\pause
	\begin{figure}[!h]
	 	\begin{tikzpicture}[->,>=stealth',shorten >=1pt,auto,node distance=4.0cm, semithick] %ein pfeil auf 0..8
  			\node(1)  [draw, rectangle,] at(0,0){
  				
  				\begin{tabular}{c| c c}
                				
                				
                				0& &	\\
					\hline
                				\vdots & &	\\
					\hline
                				7& &	\\
                				\hline

                				\vdots&  &	\\

                				\hline
                				39& &	\\
              		  		
            			\end{tabular}
  				  			
  			}  ;
 			\node(2)  [draw, rectangle,minimum width=1cm,  minimum height = 1.5cm] at(4,0){$0$} (4.6,0) node[right] {$\hdots$};
  			\node(3)  [draw, rectangle,minimum width=1cm,  minimum height = 1.5cm] at(6,0){$4$};
  			\pause
  			\path (2)	edge 			node			{}(1.1,0.55);	

		\end{tikzpicture}
		

	\end{figure}
	


\end{frame}
\begin{frame}{Distributed-Memory Hashing}
	\center
   	\begin{tabular}[h]{ c|c}
		\hline
		Funktionen				& Spezifizierung   \\
		\hline
		\pause
		find()      				& \\
		\hline
		\pause
		insert()				& \\
		\hline
		\pause
		remove()				& \\
		\hline
		\pause
		insertOrUpdate()			& \\
		\hline
		
	\end{tabular}
\end{frame}
\begin{frame}[t]{Distributed-Memory Hashing }

	Performance:\\
	
	\begin{itemize}
	
	\item Sei $o = \Omega (p\:log\:p)$ und o \textit{truly random}
	\pause
	\item  Sei $T_{all\rightarrow all}(x) $ \textit{all-to-all}  Datenaustausch und x maximale Nachrichtengöße
	\pause
	\item (Grafik)
	\pause
	\item  Zeige: max Nachrichtengröße $= O(o/p)$ 
	\pause
 	\item $O( T _{all\rightarrow all}(log\: p)) = O( T _{all\rightarrow all}(o/p))$
	\end{itemize}
\end{frame}
\begin{frame}[t]{Shared-Memory Hashing}
	\center	%eine Tabelle wird eingteilt
	\begin{itemize}
		\item	Linear probing
		\pause
		\item Offenenes Hashing
	\end{itemize}
	\vfill
	\pause
	\begin{figure}[!h]
		\begin{tikzpicture}[->,>=stealth',shorten >=1pt,auto,node distance=4.0cm, semithick]%pfeil von 0 auf gesammten Aressraum
  			\node(1)  [draw, rectangle,] at(0,0){
  				
  				\begin{tabular}{c| c c}
                				0& &	\\
                				\hline
                				& &	\\
                				\vdots&  &	\\
                				& &	\\
                				\hline
                				m-1& &	\\
            			\end{tabular}
  				  			
  			}  ;
 			\node(2)  [draw, rectangle,minimum width=1cm,  minimum height = 1.5cm] at(4,0){$0$} (4.6,0) node[right] {$\hdots$};
  			\node(3)  [draw, rectangle,minimum width=1cm,  minimum height = 1.5cm] at(6,0){$p-1$};
  			
  			\path (2)	edge 			node			{}(1.1,0);
  				

			\label{PE 0 hat Zuriff auf den kompletten Adressraum}
		\end{tikzpicture}

	\end{figure}
	\vfill


\end{frame}
\begin{frame}{Datenstruktur}
	\center
	\textbf{ Leichte Änderungen im Vergleich zum Distributet-Hashing }
	\vfill
	\pause
   	\begin{tabular}[h]{ c|c}
		\hline
		Funktionen				& Veränderung   \\
		\hline
		\pause
		find()      				& gesamter Adressraum\\
		\hline
		\pause
		insert()				& nur elementare Operationen, nächsten freien Platz automatisch finden \\
		\hline
		\pause
		remove()				& markiere zu löschendes Element  \\
							& scanne die gesamte Tabelle um echt zu löschen \\
		\hline
		\pause
		insertOrUpdate()			& nur elementare Operationen\\
		\hline
		
	\end{tabular}
	\vfill
	\pause
	Skalierbarkeit der Tabelle und Löschen nicht effizeint\\
\end{frame}


\begin{frame}{Fazit}

\begin{itemize}
	\item linear gut für kleine Mengen
	\pause
	\item distributed meisten schlechter
	\pause
	\item distributed schneller bei vielen Zugriffen auf einen Key
	\pause
	\item häufigster Zugriff:\textbf{ update}
	\pause
	\item shared schnell, da Operationen zusammen gefasst werden
\end{itemize}
	
	
\end{frame}

\end{document}